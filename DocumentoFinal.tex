% Options for packages loaded elsewhere
\PassOptionsToPackage{unicode}{hyperref}
\PassOptionsToPackage{hyphens}{url}
%
\documentclass[
]{article}
\usepackage{amsmath,amssymb}
\usepackage{iftex}
\ifPDFTeX
  \usepackage[T1]{fontenc}
  \usepackage[utf8]{inputenc}
  \usepackage{textcomp} % provide euro and other symbols
\else % if luatex or xetex
  \usepackage{unicode-math} % this also loads fontspec
  \defaultfontfeatures{Scale=MatchLowercase}
  \defaultfontfeatures[\rmfamily]{Ligatures=TeX,Scale=1}
\fi
\usepackage{lmodern}
\ifPDFTeX\else
  % xetex/luatex font selection
\fi
% Use upquote if available, for straight quotes in verbatim environments
\IfFileExists{upquote.sty}{\usepackage{upquote}}{}
\IfFileExists{microtype.sty}{% use microtype if available
  \usepackage[]{microtype}
  \UseMicrotypeSet[protrusion]{basicmath} % disable protrusion for tt fonts
}{}
\makeatletter
\@ifundefined{KOMAClassName}{% if non-KOMA class
  \IfFileExists{parskip.sty}{%
    \usepackage{parskip}
  }{% else
    \setlength{\parindent}{0pt}
    \setlength{\parskip}{6pt plus 2pt minus 1pt}}
}{% if KOMA class
  \KOMAoptions{parskip=half}}
\makeatother
\usepackage{xcolor}
\usepackage[margin=1in]{geometry}
\usepackage{longtable,booktabs,array}
\usepackage{calc} % for calculating minipage widths
% Correct order of tables after \paragraph or \subparagraph
\usepackage{etoolbox}
\makeatletter
\patchcmd\longtable{\par}{\if@noskipsec\mbox{}\fi\par}{}{}
\makeatother
% Allow footnotes in longtable head/foot
\IfFileExists{footnotehyper.sty}{\usepackage{footnotehyper}}{\usepackage{footnote}}
\makesavenoteenv{longtable}
\usepackage{graphicx}
\makeatletter
\def\maxwidth{\ifdim\Gin@nat@width>\linewidth\linewidth\else\Gin@nat@width\fi}
\def\maxheight{\ifdim\Gin@nat@height>\textheight\textheight\else\Gin@nat@height\fi}
\makeatother
% Scale images if necessary, so that they will not overflow the page
% margins by default, and it is still possible to overwrite the defaults
% using explicit options in \includegraphics[width, height, ...]{}
\setkeys{Gin}{width=\maxwidth,height=\maxheight,keepaspectratio}
% Set default figure placement to htbp
\makeatletter
\def\fps@figure{htbp}
\makeatother
\setlength{\emergencystretch}{3em} % prevent overfull lines
\providecommand{\tightlist}{%
  \setlength{\itemsep}{0pt}\setlength{\parskip}{0pt}}
\setcounter{secnumdepth}{-\maxdimen} % remove section numbering
\ifLuaTeX
  \usepackage{selnolig}  % disable illegal ligatures
\fi
\IfFileExists{bookmark.sty}{\usepackage{bookmark}}{\usepackage{hyperref}}
\IfFileExists{xurl.sty}{\usepackage{xurl}}{} % add URL line breaks if available
\urlstyle{same}
\hypersetup{
  pdftitle={Informe\_Final},
  pdfauthor={Anshela Castillo, Milka Chasi, Marilyn Suquillo},
  hidelinks,
  pdfcreator={LaTeX via pandoc}}

\title{Informe\_Final}
\author{Anshela Castillo, Milka Chasi, Marilyn Suquillo}
\date{2023-08-06}

\begin{document}
\maketitle

\{=html\}

\{r setup, include=FALSE\} knitr::opts\_chunk\$set(echo = TRUE)

\hypertarget{proyecto-final-del-muxf3dulo---introducciuxf3n-a-programaciuxf3n-en-r}{%
\subsection{Proyecto final del módulo - Introducción a programación en
R}\label{proyecto-final-del-muxf3dulo---introducciuxf3n-a-programaciuxf3n-en-r}}

\hypertarget{introducciuxf3n}{%
\subsubsection{Introducción}\label{introducciuxf3n}}

Una visión cuantitativa de la salud financiera de las empresas puede ser
proporcionada empleando los ratios o indicadores financieros. Los
indicadores financieros son herramientas clave en el análisis económico
de las empresas porque permiten evaluar su eficiencia operativa, la
liquidez, rentabilidad y solidez. Estas herramientas, básicamente, son
relaciones matemáticas entre diferentes elementos de los estados
financieros. Entre los diferentes tipos de indicadores tenemos 1) los
indicadores de liquidez, que evalúan la capacidad de las empresas para
cumplir con sus obligaciones a corto plazo: es decir, evalúan si la
empresa tiene suficientes recursos disponibles para pagar sus deudas y
gastos operativos. Además, reflejan la disponibilidad de activos
líquidos en relación con los pasivos de corto plazo.; 2) los indicadores
de gestión, que miden la eficiencia con la que una empresa utiliza sus
recursos para generar ingresos y beneficios para identificar áreas en
las que pueden mejorar su eficiencia operativa y optimizar recursos. Un
ejemplo es el margen de utilidad neta; 3) los indicadores de solvencia,
que evalúan la capacidad de una empresa para cumplir con sus
obligaciones financieras a largo plazo teniendo en cuenta sus activos y
pasivos. Estos indicadores reflejan la estructura de capital de la
empresa y su nivel de endeudamiento, y son cruciales para determinar si
una empresa tiene la capacidad para pagar sus deudas a largo plazo y
mantener su estabilidad financiera; 4) los indicadores de rentabilidad,
que miden la capacidad de una empresa para generar beneficios en
relación con sus ingresos y activos. El ROI es un ejemplo común que
muestra el rendimiento de la inversión en términos de ganancias
generadas (Garcés, 2019; Lee, 2023, Párraga et al., 2021). La
importancia de un análisis financiero es la aplicación de estrategias
que permitan visualizar el nivel de liquidez, solvencia, endeudamiento y
rentabilidad en la actividad empresarial, evaluando el rendimiento de un
negocio. Esta herramienta facilita la toma de decisiones. En la
actualidad, los análisis de finanzas, empleando indicadores financieros,
pueden ser calculados a través de la Ciencia de datos que se sirve de
software o lenguajes de programación para su análisis. Contar con
información detallada y analizada de estos indicadores, permite a las
empresas tomar decisiones claras y acertadas en un plan de acción que
permita identificar sus puntos fuertes y débiles empresariales comparada
con otros negocios (Párraga et al., 2021).

\begin{center}\rule{0.5\linewidth}{0.5pt}\end{center}

\hypertarget{objetivo}{%
\subsubsection{Objetivo}\label{objetivo}}

\hypertarget{el-presente-trabajo-tiene-como-objetivo-calcular-los-indicadores-financieros-de-varias-empresas-empleando-el-lenguaje-de-programaciuxf3n-r-para-analizar-su-impacto-en-micro-pequeuxf1as-y-grandes-empresas.}{%
\subsection{El presente trabajo tiene como objetivo calcular los
indicadores financieros de varias empresas empleando el lenguaje de
programación R para analizar su impacto en micro, pequeñas y grandes
empresas.}\label{el-presente-trabajo-tiene-como-objetivo-calcular-los-indicadores-financieros-de-varias-empresas-empleando-el-lenguaje-de-programaciuxf3n-r-para-analizar-su-impacto-en-micro-pequeuxf1as-y-grandes-empresas.}}

\hypertarget{preguntas-de-investigaciuxf3n}{%
\subsubsection{Preguntas de
investigación}\label{preguntas-de-investigaciuxf3n}}

\begin{enumerate}
\def\labelenumi{\arabic{enumi}.}
\tightlist
\item
  ¿El endeudamiento del activo fue mayor en empresas micro + pequeñas
  vs.~grandes?
\item
  ¿La liquidez por tipo de compañía es diferente entre aquellas empresas
  que tienen más de 60 trabajadores directos y que cuenta con 100 a 800
  trabajadores administrativos?
\item
  ¿Cuáles son las 10 empresas con mayor apalancamiento?
\end{enumerate}

\begin{center}\rule{0.5\linewidth}{0.5pt}\end{center}

\hypertarget{descripciuxf3n-de-los-datos}{%
\subsubsection{Descripción de los
datos}\label{descripciuxf3n-de-los-datos}}

Creación del repositorio El repositorio se creó en GITHUB con el nombre
``Proyecto Final R'', el mismo que consta de carpetas específicas: 1)
Datos, donde se alojó los datos que posteriormente se usaron para el
desarrollo del presente trabajo, 2) Script, donde se guardó el código
programado en R y 3) Plots, donde se guardó las 3 gráficas generadas que
responden a las preguntas de investigación. Posterior a la creación, se
compartió y se clonó dos veces para trabajar de manera remota en el
mismo.

Escritura del código en R. Una vez que se instalaron y se cargaron los
paquetes (tidyverse, openxlsx, readr, dplyr, readxl, ggplot2) se
procedió a cargar la data (balances\_2014.xlsx: 47033 observaciones de
347 variables) que fue almacenada en el objeto ``empresas'' y
conjuntamente fue convertida a un tibble. A continuación, se verificó si
existen datos faltantes. Para la limpieza de la data se ejecutó un bucle
que A continuación, se realizó un Re shape para crear o construir la
data base, la cual fue guardada en el objeto ``empresas. La construcción
consta de 2 fases: calcular los indicadores financieros, seleccionar las
columnas necesarias y finalmente renombrarlas. Los verbos de dplyr que
se usaron para completar estas fases select y mutate, junto con rename y
operaciones matemáticas. Esta base fue la base a ser empleada para la
construcción de las gráficas y responder las preguntas.

Preguntas.\\
Para la pregunta uno, fue clave emplear la función is.finite porque es
util cuando se trabaja con conjuntos de datos que contengan valores
infitos o NA, cómo es el caso actual en la data ``empresas''. Además,
tambien se eliminó los valores NA empleando na.rm =TRUE.\\
Para la pregunta dos, del mismo modo se usó las funciones empleadas
anteriormete y, en adición, se emplearon funciones cómo group by y
filter. Para la pregunta tres, del mismo modo se empleó la función
is.finite, arrange y head. En las tres preguntas se empleó la fúncion
ggplot con sus respectivos argumentos para obtener las gráficas
respectivas.

\begin{longtable}[]{@{}
  >{\raggedright\arraybackslash}p{(\columnwidth - 0\tabcolsep) * \real{0.7639}}@{}}
\toprule\noalign{}
\endhead
\bottomrule\noalign{}
\endlastfoot
\#\#\# Análisis \emph{Parte 1 - Preguntas de investigación} Primero, a
la data base obtenida anteriormente se le añadió las columnas necesarias
Pregunta 1 ¿El endeudamiento del activo fue mayor en empresas micro +
pequeñas vs.~grandes? \{r message=FALSE, warning=FALSE\}
PM\textless-empresas \%\textgreater\%
select(tamanio,Endeudamiento\_del\_activo) \%\textgreater\%
filter(tamanio==``PEQUEÑA'' \textbar{} tamanio==``MICRO'')
PM\_limpio\textless-PM {[}
is.finite(PM\(Endeudamiento_del_activo), ] E_activo_PM<-sum(PM_limpio\)Endeudamiento\_del\_activo,
na.rm = TRUE) \\
En la gráfica se puede observar que entre las empresas micros y pequeñas
poseen un total de endeudamiento del activo alto; en proporción relativa
total esto representa alrededor del 92\% de empresas con mayor
endeudamiento del activo. Sin embargo, cuando analizamos las empresas
grandes no alcanzan a cubrir ni la cuarta parte de endeudamiento del
activo de las empresas micro más pequeñas. En cifras esto sería
aproximadamente el 8\% de endeudamiento del activo en empresas
grandes. \\
Pregunta 2 ¿La liquidez por tipo de compañía es diferente entre aquellas
empresas que tienen más de 60 trabajadores directos y que cuenta con 100
a 800 trabajadores administrativos? \{r message=FALSE, warning=FALSE\}
PM\textless-empresas \%\textgreater\%
select(tamanio,Endeudamiento\_del\_activo) \%\textgreater\%
filter(tamanio==``PEQUEÑA'' \textbar{} tamanio==``MICRO'')
PM\_limpio\textless-PM {[}
is.finite(PM\(Endeudamiento_del_activo), ] E_activo_PM<-sum(PM_limpio\)Endeudamiento\_del\_activo,
na.rm = TRUE) \\
G\textless-empresas \%\textgreater\%
select(tamanio,Endeudamiento\_del\_activo) \%\textgreater\%
filter(tamanio==``GRANDE'')
E\_activo\_G\textless-sum(G\$Endeudamiento\_del\_activo, na.rm =
TRUE) \\
RESULTADOS\textless-data.frame( Tipo\_empresa = c(``Micro + Pequeñas'',
``Grandes''), Endeudamiento=c(E\_activo\_PM,E\_activo\_G) ) \\
ggplot(RESULTADOS, aes(x = Tipo\_empresa, y = Endeudamiento)) +
geom\_bar(stat = ``identity'', fill= ``blue'') + labs(title =
``Endeudamiento del activo en empresas según el tamaño'', x = ``Tamaño
empresa'', y = ``Endeudamiento del activo'') + theme\_minimal() \\
En la gráfica se puede observar una gran diferencia. Las empresas que
cuenta con 100-800 trabajadores adminitrativos tienen mayor liquidez, en
comparación con aquellas que poseen 60 trabajadores directos. \\
Pregunta 3 ¿Cuáles son las 10 empresas con mayor apalancamiento? \{r
message=FALSE, warning=FALSE\} TOP\_APAL\textless-empresas
\%\textgreater\% select(Empresas,Apalancamiento)
TOP\_APAL\_limpio\textless- TOP\_APAL{[}
is.finite(TOP\_APAL\$Apalancamiento), {]} \\
TOP\_ordenados\textless-TOP\_APAL\_limpio \%\textgreater\%
arrange(desc(Apalancamiento))
TOP\_10\textless-head(TOP\_ordenados,10) \\
ggplot(TOP\_10, aes(x = Apalancamiento, y =
reorder(Empresas,Apalancamiento))) + geom\_bar(stat = ``identity'',
fill= ``red'') + labs(title = ``10 empresas mejor apalancadas'', x =
``Apalancamiento'', y = ``Empresas'') + theme(axis.text.x =
element\_text(angle = 45, hjust = 1)) \\
Se puede observar que la emprea con mayor apalancamiento es Adelca del
Litoral S. A, seguida por Faribal Holding Corp, Hiroky S. A, y
Megatropic S.A. Dentro del top 10, la empresa con menor apalancamiento
es Ecuadesk SA. \\
\emph{Parte 2 - Tareas específicas} \\
1. Utilizando los datos de balance 2014 genera un tibble que denomines
empresas El tibble empresas, es una base de datos que alberga 46578
observaciones de 347 variables. La misma que puede ser visualizada en la
consola de R al ejecutar el código: \{r message=FALSE, warning=FALSE\}
view(``empresas) \\
2. Crea una tabla resumiendo el número total de empresas pro actividad
económica y por actividad económica por cantón (dataframe o tibble en tu
script) \{r message=FALSE, warning=FALSE\} data\_2\textless-
as\_tibble(read.xlsx(``Datos/ciiu.xlsx'')) data\_2\textless- data\_2
\%\textgreater\% filter(CODIGO==``A'' \textbar{} CODIGO==``B''
\textbar{} CODIGO==``C''\textbar{}
CODIGO==``D''\textbar CODIGO==``E''\textbar{}
CODIGO==``F''\textbar CODIGO==``G''\textbar CODIGO==``H''\textbar CODIGO==``I''\textbar CODIGO==``J''\textbar CODIGO==``K''\textbar{}
CODIGO==``L''\textbar CODIGO==``M''\textbar CODIGO==``N''\textbar CODIGO==``O''\textbar CODIGO==``P''\textbar CODIGO==``Q''\textbar{}
CODIGO==``R''\textbar CODIGO==``S''\textbar CODIGO==``T''\textbar CODIGO==``U''\textbar CODIGO==``Z'')
data\_2\textless-data\_2 \%\textgreater\%select(CODIGO,DESCRIPCION) \\
data\_3\textless-empresas \%\textgreater\% select(Actividad\_económica)
tabla1\textless-data\_3 \%\textgreater\% group\_by(Actividad\_económica)
\%\textgreater\% summarise(Ntotal\_emp\_Actividad\_eco=n())
\%\textgreater\%
left\_join(data\_2,by=c(``Actividad\_económica''=``CODIGO''))
tabla1\textless-select(tabla1,Actividad\_económica,DESCRIPCION,Ntotal\_emp\_Actividad\_eco)
tabla1 \\
tabla2\textless-empresas\%\textgreater\%
group\_by(Actividad\_económica,Cantón) \%\textgreater\%
summarise(Ntotal\_empresas\_ecoycanton=n()) \%\textgreater\%
left\_join(data\_2,by=c(``Actividad\_económica''=``CODIGO''))
tabla2\textless-select(tabla2,Actividad\_económica,DESCRIPCION,Cantón,Ntotal\_empresas\_ecoycanton) \\
3. Gráficamente muestra el comparativo de los indicadores financieros de
liquidez y solvencia por Status y provincia. \{r message=FALSE,
warning=FALSE\} ggplot(empresas, aes(x =Provincia, y =
Liquidez\_corriente,fill=Status)) + geom\_bar(stat = ``summary'',
position = ``stack'')+ labs(title = ``liquidez\_corriente por Status y
Provincia'', x = ``Provincia'', y = ``Liquidez\_corriente'') +
theme(axis.text.x = element\_text(angle = 45, hjust = 1)) \\
ggplot(empresas, aes(x =Provincia, y =
Endeudamiento\_del\_activo,fill=Status)) + geom\_bar(stat = ``summary'',
position = ``stack'') + labs(title = ``Endeudamiento del activo por
Status y Provincia'', x = ``Provincia'', y = ``Endeudamiento del
activo'') + theme(axis.text.x = element\_text(angle = 45, hjust = 1)) \\
ggplot(empresas, aes(x =Provincia, y =
Endeudamiento\_patrimonial,fill=Status)) + geom\_bar(stat = ``summary'',
position = ``stack'') + labs(title = ``Endeudamiento patrimonial por
Status y Provincia'', x = ``Provincia'', y = ``Endeudamiento
patrimonial'') + theme(axis.text.x = element\_text(angle = 45, hjust =
1)) \\
ggplot(empresas, aes(x =Provincia, y =
Endeudamiento\_del\_Activo\_Fijo,fill=Status)) + geom\_bar(stat =
``summary'', position = ``stack'') + labs(title =
``Endeudamiento\_del\_Activo\_Fijo por Status y Provincia'', x =
``Provincia'', y = ``Endeudamiento\_del\_Activo\_Fijo'') +
theme(axis.text.x = element\_text(angle = 45, hjust = 1)) \\
ggplot(empresas, aes(x =Provincia, y = Apalancamiento,fill=Status)) +
geom\_bar(stat = ``summary'', position = ``stack'') + labs(title =
``Comparativo de Apalancamiento por Status y Provincia'', x =
``Provincia'', y = ``Apalancamiento'') + theme(axis.text.x =
element\_text(angle = 45, hjust = 1)) 4. Gráficamente muestra el
comparativo de los indicadores financieros de liquidez y sovencia por
tipo de empresa. \{r message=FALSE, warning=FALSE\} ggplot(empresas,
aes(x = Tipo\_de\_empresa)) + geom\_line(aes(y = Liquidez\_corriente,
group = 1, color = ``Liquidez Corriente''), stat = ``summary'', fun =
``mean'', position = ``dodge'', size = 1) + geom\_point(aes(y =
Liquidez\_corriente, group = 1, color = ``Liquidez Corriente''), stat =
``summary'', fun = ``mean'', position = ``dodge'', size = 3) +
geom\_line(aes(y = Endeudamiento\_del\_activo, group = 1, color =
``Endeudamiento del Activo''), stat = ``summary'', fun = ``mean'',
position = ``dodge'', size = 1) + geom\_point(aes(y =
Endeudamiento\_del\_activo, group = 1, color = ``Endeudamiento del
Activo''), stat = ``summary'', fun = ``mean'', position = ``dodge'',
size = 3)+ geom\_line(aes(y = Endeudamiento\_patrimonial, group = 1,
color = ``Endeudamiento Patrimonial''), stat = ``summary'', fun =
``mean'', position = ``dodge'', size = 1) + geom\_line(aes(y =
Endeudamiento\_del\_Activo\_Fijo, group = 1, color = ``Endeudamiento del
Activo Fijo''), stat = ``summary'', fun = ``mean'', position =
``dodge'', size = 1) + geom\_line(aes(y = Apalancamiento, group = 1,
color = ``Apalancamiento''), stat = ``summary'', fun = ``mean'',
position = ``dodge'', size = 1) + geom\_point(aes(y =
Endeudamiento\_patrimonial, group = 1, color = ``Endeudamiento
Patrimonial''), stat = ``summary'', fun = ``mean'', position =
``dodge'', size = 3) + geom\_point(aes(y =
Endeudamiento\_del\_Activo\_Fijo, group = 1, color = ``Endeudamiento del
Activo Fijo''), stat = ``summary'', fun = ``mean'', position =
``dodge'', size = 3) + geom\_point(aes(y = Apalancamiento, group = 1,
color = ``Apalancamiento''), stat = ``summary'', fun = ``mean'',
position = ``dodge'', size = 3)+ scale\_color\_manual(values =
c(``Liquidez Corriente'' = ``blue'', ``Endeudamiento del Activo'' =
``red'', ``Endeudamiento Patrimonial'' = ``green'', ``Endeudamiento del
Activo Fijo'' = ``orange'', ``Apalancamiento'' = ``purple'')) +
theme\_minimal() + labs(title = ``Comparativo de los indicadores
financieros de liquidez y solvencia por tipo de empresa'', x = ``Tipo de
Empresa'', y = ``Valor'', color = ``Indicadores Financieros'') + \#
Cambiar el título de la leyenda theme(axis.text.x = element\_text(angle
= 45, hjust = 1)) \\
\end{longtable}

\hypertarget{conclusiuxf3n}{%
\subsubsection{Conclusión}\label{conclusiuxf3n}}

Los indicadores financieros calculado para las empresas nos permite
concluir que son micro y pequeñas empresas las que tienen una alta
capacidad de endeudamiento, incurriendo en un mayor riesgo, en
comparación con las empresas grandes; es decir, los activos de las micro
y pequeñas empresas se financian a través de deuda. Además, son las
grandes empresas con mayor número de trabajadores las que tienen mayor
liquidez. Finalmente, empresas grandes como ADELCA, una empresa
siderúrgica, supera altamente a las demás empresas analizadas en
referencia al apalancamiento.

\end{document}
